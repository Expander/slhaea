\documentclass[
  11pt,
  ]{revtex4-1}

\usepackage[utf8x]{inputenc}
\usepackage[T1]{fontenc}
\usepackage{bm}
\usepackage[
  pdftex,
  colorlinks=true,
  bookmarksopen=true,
  bookmarksopenlevel=1,
  ]{hyperref}
\usepackage[english]{babel}
\usepackage[babel=true]{microtype}
\usepackage{lmodern}
\linespread{1.05}
\usepackage{listings}
\lstloadlanguages{[ISO]C++}
\lstset{
  language=[ISO]C++,
  basicstyle=\ttfamily,
  commentstyle=\rmfamily,
  xleftmargin=.06\linewidth,
  xrightmargin=.06\linewidth,
  columns=fullflexible,
}


\begin{document}

\title{SLHAea - another SUSY Les Houches Accord input/output library}
\date{\today}

\author{F. S. Thomas}
\email{fthomas@physik.uni-wuerzburg.de}
\affiliation{Institut für Theoretische Physik und Astrophysik,
  Universität Würzburg, Am Hubland, 97074 Würzburg, Germany}

\begin{abstract}
\end{abstract}

\maketitle
\tableofcontents

\section{Introduction\label{sec:introduction}}

\section{Data structures\label{sec:data_structures}}

\subsection{Line\label{ssec:line}}

\subsection{Block\label{ssec:block}}

\subsection{Coll\label{ssec:coll}}

\subsection{Key\label{ssec:key}}

\section{Examples\label{sec:examples}}

\section{Summary\label{sec:summary}}

<introducing example>
<data structures (or containers)>
 <similarity to STL containers>
 <Line>
 <Block>
 <Coll>
 <Key>

<real world examples>

<compiling, compiling examples, tested with different compilers,
std conformance>

<summary>

\begin{lstlisting}
#include <fstream>
#include <slhaea.h>

int main()
{
  std::ifstream ifs("SLHA.par"); // comment
  SLHAea::Coll input(ifs); /* comment */
}
\end{lstlisting}


\bibliographystyle{apsrev4-1long}
\bibliography{slhaea}
\nocite{*}

\end{document}
